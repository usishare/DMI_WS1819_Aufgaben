% layout and global options
\documentclass
[
  draft    = true,
  fontsize = 11pt,
  parskip  = half-,
  BCOR     = 0pt,
  DIV      = 11,
  ngerman,
  dvipsnames
]
{scrartcl}

% default packages
\usepackage[utf8]{inputenc}
\usepackage[T1]{fontenc}
\usepackage{lmodern}
\usepackage{babel}
% extra packages
\usepackage{amsmath}
\usepackage{amssymb}
\usepackage{enumerate}
\usepackage{graphicx}
\usepackage{ifthen}
\usepackage{siunitx}
\usepackage{tikz}
\usepackage{url}

% basic calculations in TikZ
\usetikzlibrary{calc}

% use comma as decimal separator
\sisetup{locale=DE, group-minimum-digits=4}

% create links
\usepackage[draft=false]{hyperref}

% ------------------------------------------------------------------------------
\begin{document}
% ------------------------------------------------------------------------------

\tableofcontents

% -----------------------------------------
\section{Aussagenlogik und Prädikatenlogik}
% -----------------------------------------

% ----------------------------
\paragraph{Übung 1: Aufgabe 1}
% ----------------------------
Sind die folgenden Aussagen im logischen Sinne wahr oder falsch?

% ----------------------------
\paragraph{Übung 2: Aufgabe 3}
% ----------------------------
Drücken Sie die folgenden Aussagen in Prädikatenlogik über den natürlichen
Zahlen ohne Null $(\mathbb{N}_{>0})$ aus. Dabei stehen Ihnen als zwei stellige
Funktionssymbole $+$ und $\cdot$ sowie das zweistellige Prädikat $=$ mit jeweils
den üblichen Bedeutungen zur Verfügung.

% ------------------------
\section{Beweisprinzipien}
% ------------------------

% ----------------------------
\paragraph{Übung 5: Aufgabe 1}
% ----------------------------
Beweisen Sie die folgenden Aussagen mittels vollständiger Induktion über
$n$

% -------------------------------------
\section{Mengentheoretische Grundlagen}
% -------------------------------------

% ----------------------------
\paragraph{Übung 1: Aufgabe 2}
% ----------------------------
Bestimmen Sie die folgenden Mengen

% ----------------------------
\paragraph{Übung 2: Aufgabe 1}
% ----------------------------
Bestimmen Sie die folgenden Mengen

% ----------------------------
\paragraph{Übung 3: Aufgabe 1}
% ----------------------------
Bestimmen Sie, ob es sich bei den folgenden Relationen um Funktionen
handelt. Geben Sie außerdem fur die gefundenen Funktionen das Bild der
Funktion an.

% ----------------------------
\paragraph{Übung 3: Aufgabe 2}
% ----------------------------
Geben Sie eine Bijektion von A nach B an.

% ----------------------------
\paragraph{Übung 3: Aufgabe 3}
% ----------------------------
Bestimmen Sie, ob folgende Funktionen injektiv bzw. surjektiv sind

% ----------------------------
\paragraph{Übung 4: Aufgabe 1}
% ----------------------------
Bestimmen Sie, ob folgende Relationen reflexiv, irreflexiv, symmetrisch,
antisymmetrisch bzw. transitiv sind

% ----------------------------
\paragraph{Übung 4: Aufgabe 2}
% ----------------------------
Bestimmen Sie die Äquivalenzklassen der folgenden Relationen

% --------------------
\section{Kombinatorik}
% --------------------

% ------------------------------
\subsection{Abzählen von Mengen}
% ------------------------------

% ----------------------------
\paragraph{Übung 6: Aufgabe 1}
% ----------------------------
\begin{enumerate}[a)]
  \item Bestimmen Sie die Menge aller möglichen Wörter der Länge $n$ über einem Alphabet mit $k$ Buchstaben.
  \item An einem Marathon nehmen 20 Läufer teil. In wie vielen verschiedenen Reihenfolgen können die Läufer das Ziel erreichen?
  \item Wie viele verschiedene mögliche Zahlenkombinationen können bei einer Lottoziehung (6 aus 49) gezogen werden?
  \item Wie viele verschiedene Notenverteilungen können entstehen, wenn bei einer Klausur 20 Studenten mitschreiben?
  \item Aus einer Schulklasse von 23 Schülern soll eine Gruppe von 5 Schülern zum Direktor geschickt werden. Auf wie viele Arten kann diese Abordnung gebildet werden?
  \item In einem Zimmer gibt es 5 Lampen, die unabhängig voneinander aus- und eingeschaltet werden können. Wie viele Arten der Beleuchtung gibt es insgesamt?
  \item In einer Liga spielen 7 Mannschaften. Wie viele mögliche Tabellenkonstellationen gibt es?
  \item Wie viele natürliche Zahlen können als Produkt von 10 Faktoren aus den Zahlen 1,2,3 und 4 dargestellt werden?
\end{enumerate}

% ----------------------------
\paragraph{Übung 6: Aufgabe 2}
% ----------------------------
Bestimmen Sie
\begin{equation*}
  \text{a) }\binom{6}{2}
  \qquad
  \text{und}
  \qquad
  \text{b) }\binom{n}{2}\text{ für }n\in\mathbb{N}
\end{equation*}

% -------------------------------
\subsection{Einfache Identitäten}
% -------------------------------

% ----------------------------
\paragraph{Übung 6: Aufgabe 2}
% ----------------------------
Zeigen Sie, dass für alle $n,k\in\mathbb{N}$ folgende Gleichung gilt:
\begin{equation*}
  k\cdot\binom{n}{k}=n\cdot\binom{n-1}{k-1}
\end{equation*}

% ----------------------------
\paragraph{Übung 7: Aufgabe 1}
% ----------------------------
Zeigen Sie, dass für alle $n\in\mathbb{N}$ folgende Gleichung gilt:
\begin{equation*}
  \sum_{i=0}^{n}\sum_{j=0}^{i}\binom{i}{j}=2^{n+1}-1
\end{equation*}

% ----------------------------------
\subsection{Der Binomische Lehrsatz}
% ----------------------------------

% ----------------------------
\paragraph{Übung 7: Aufgabe 2}
% ----------------------------
\begin{enumerate}[a)]
  \item Lösen Sie die folgende Gleichung mit Hilfe des binomischen Lehrsatzes:
        \begin{equation*}
          x^3++3x^2+3x+1=8
        \end{equation*}
  \item Zeigen Sie, dass für $n\geq1$ folgende Gleichung erfüllt ist:
        \begin{equation*}
          \sum_{k=0}^{n}\binom{n}{k}(-1)^k=0
        \end{equation*}
\end{enumerate}

% ----------------------
\section{Graphentheorie}
% ----------------------

% ------------------------
\subsection{Grundbegriffe}
% ------------------------

% ----------------------------
\paragraph{Übung 7: Aufgabe 3}
% ----------------------------
Gegeben sei ein ungerichteter Graph
\begin{equation*}
  G=(\{1,2,3,4,5\},\{\{1,2\},\{1,4\},\{1,5\},\{2,3\},\{2,5\},\{3,4\}\}).
\end{equation*}

% ----------------------------
\paragraph{Übung 7: Aufgabe 4}
% ----------------------------
Beweisen Sie, dass jeder ungerichtete Graph $G=(V,E)$ mit $|V|\geq2$
mindestens 2 Knoten mit gleichem Grad hat!

% ----------------------------
\paragraph{Übung 7: Aufgabe 5}
% ----------------------------
Wie viele Graphen mit $n$ Knoten gibt es?

% ----------------------------
\paragraph{Übung 7: Aufgabe 6}
% ----------------------------
Beweisen Sie: $C_n$ ist bipartit genau dann, wenn $n$ gerade ist.

% ----------------------------
\paragraph{Übung 8: Aufgabe 3}
% ----------------------------
Sei $G$ ein Baum mit 6 Knoten. Wie viele Blätter kann $G$ enthalten?

% ----------------------------
\paragraph{Übung 8: Aufgabe 4}
% ----------------------------
Wie viele Kreise der Länge $r$ enthält der vollständige Graph $K_n$?

% ----------------------------
\paragraph{Übung 9: Aufgabe 5}
% ----------------------------
Sei $G$ ein Graph mit $n$ Knoten.
\begin{enumerate}[a)]
  \item Was ist die kleinste Anzahl an Kanten $m$, die man braucht, so dass $G$ zusammenhängend ist?
  \item Wie viele Kanten muss $G$ mindestens haben, so dass $G$ in jedem Fall zusammenhängend ist?
\end{enumerate}

% --------------------------
\subsection{Planare Graphen}
% --------------------------

% ----------------------------
\paragraph{Übung 8: Aufgabe 1}
% ----------------------------
Zeichnen Sie die folgenden Graphen planar

% ----------------------------
\paragraph{Übung 8: Aufgabe 2}
% ----------------------------
Sei $G_{n,m}$ der Graph bestehend aus $n\cdot m$ Quadraten, wobei jeweils
$n$ Quadrate untereinander und $m$ Quadrate nebeneinander liegen.
Zusätzlich sind die beiden diagonal gegenüber liegenden Punkte in einem
Quadrat miteinander verbunden.

% --------------------
\subsection{Färbungen}
% --------------------

% ----------------------------
\paragraph{Übung 9: Aufgabe 2}
% ----------------------------
Beweisen Sie die folgenden Aussagen:

% ----------------------------
\paragraph{Übung 9: Aufgabe 3}
% ----------------------------
Gegeben sei folgender Graph G:
Bestimmen Sie $\chi(G)$ und geben Sie eine 4-Färbung an.

% ----------------------------
\paragraph{Übung 9: Aufgabe 4}
% ----------------------------
Beweisen Sie: Für einen Graphen mit m Kanten gilt
\begin{equation*}
  \ldots
\end{equation*}
Hinweis: Nehmen Sie an, ihr Graph hat $\chi(G)$ Farbklassen. Was können
Sie dann für die Anzahl der Kanten zwischen den Farbklassen folgern?

% --------------------
\subsection{Matchings}
% --------------------

% -----------------------------
\paragraph{Übung 10: Aufgabe 1}
% -----------------------------
Gegeben sei folgender Graph und das Matching
\begin{equation*}
  M=\{\{h,f\},\{c,e\},\{a,d\}\}
\end{equation*}

% -----------------------------
\paragraph{Übung 10: Aufgabe 2}
% -----------------------------
Bestimmen Sie die Anzahl der perfekten Matchings in folgenden Graphen:

% -----------------------------------
\subsection{Euler- und Hamiltonpfade}
% -----------------------------------

% -----------------------------
\paragraph{Übung 10: Aufgabe 3}
% -----------------------------
Zeichnen Sie den Graph $G=(V,E)$
\begin{enumerate}[a)]
  \item Enthält $G$ einen Eulerweg bzw. einen Eulerkreis?
  \item Enthält $G'$ einen Eulerweg bzw. einen Eulerkreis?
  \item Enthält $G''$ einen Eulerweg bzw. einen Eulerkreis?
\end{enumerate}

% -----------------------------
\paragraph{Übung 10: Aufgabe 4}
% -----------------------------
Beweisen Sie, dass ein Graph $G=(V,E)$ mit $d_u=2$ einen Eulerweg, aber
keinen Eulerkreis hat.

% -----------------------------
\paragraph{Übung 10: Aufgabe 5}
% -----------------------------
\begin{enumerate}[a)]
  \item Beweisen Sie: $K_n$ besitzt für $n\geq3$ einen Hamiltonkreis.
  \item Sei $G$ ein Graph mit $n\geq3$ Knoten. Wie viele Kanten muss $G$
        mindestens enthalten, damit $G$ auf jeden Fall einen Hamiltonkreis
        besitzt?
\end{enumerate}

% ---------------------------
\section{Monoide und Gruppen}
% ---------------------------

% -----------------------------
\paragraph{Übung 11: Aufgabe 1}
% -----------------------------
Beweisen Sie: Ist $(G,\circ)$ eine Gruppe und $a,b\in G$, so gibt es ein
eindeutiges $c\in G$ mit $a\circ c=b$.

% -----------------------------
\paragraph{Übung 11: Aufgabe 2}
% -----------------------------
Beweisen oder widerlegen Sie die folgenden Aussagen: In jeder Gruppe
$(G,\circ)$ mit neutralem Element $e$ gilt für alle $a,b\in G$

% -----------------------------
\paragraph{Übung 11: Aufgabe 3}
% -----------------------------
Zeigen Sie, dass es eine Gruppe $G$ und Elemente $a,b\in G$ gibt, so
dass die Gleichung $(ab)^{-1}=a^{-1}b^{-1}$ nicht erfüllt ist.

% -----------------------------
\paragraph{Übung 11: Aufgabe 4}
% -----------------------------
Geben Sie die Verknüpfungstabellen der folgenden Monoide an und bestimmen
Sie, welches Monoid eine Gruppe ist:

% -----------------------------
\paragraph{Übung 11: Aufgabe 7}
% -----------------------------
\begin{enumerate}[a)]
  \item Geben Sie alle Untergruppen der folgenden Gruppen an:
        \begin{enumerate}[i)]
          \item $S_3$
          \item $(\mathbb{Z}_8,+)$
        \end{enumerate}
  \item Finden Sie, falls möglich, zu den beiden Gruppen je zwei
        Untergruppen, deren Vereinigung keine Untergruppe ist.
\end{enumerate}

% -----------------------------
\paragraph{Übung 11: Aufgabe 8}
% -----------------------------
Zeigen Sie, dass jede Untergruppe einer unendlichen zyklischen Gruppe selbst zyklisch ist.

% -----------------------------
\paragraph{Übung 12: Aufgabe 1}
% -----------------------------
Es seien $K$ und $L$ zwei Untergruppen einer endlichen Gruppe $(G,\circ)$.
Beweisen oder widerlegen Sie:
\begin{enumerate}[a)]
  \item
  \item
\end{enumerate}

% -----------------------------
\paragraph{Übung 12: Aufgabe 2}
% -----------------------------
Sei $(G, \circ)$ eine Gruppe und $U$ eine Untergruppe von $G$. Zeigen Sie, dass
$U$ genau dann ein Normalteiler ist, wenn für alle $a\in G$ die Gleichung
$a\circ U=U\circ a$ gilt, also wenn Links- und Rechtsnebenklassen von $U$
übereinstimmen.

% -----------------------------
\paragraph{Übung 12: Aufgabe 3}
% -----------------------------
Zeigen Sie, dass für alle $n\in\mathbb{N}\setminus\{0\}$ die Gruppe
$(Z,+)/n\mathbb{Z}$ isomorph ist zu $(Z_n, +_n)$.

% -----------------------------
\paragraph{Übung 12: Aufgabe 5}
% -----------------------------
Gegeben sei die Gruppe $G=(\mathbb{Z},+)$, deren Untergruppe $U=4\mathbb{Z}$
und die Abbildung $\varphi:G/U\rightarrow(\mathbb{Z}_2 ,+_2)$ mit
$\varphi(a+4\mathbb{Z})=a\bmod2$ für $a\in\mathbb{Z}$. Zeigen Sie, dass
$\varphi$
\begin{enumerate}[a)]
  \item
  \item
  \item
\end{enumerate}

% ---------------------
\section{Zahlentheorie}
% ---------------------

% -----------------------------
\paragraph{Übung 11: Aufgabe 5}
% -----------------------------
Berechnen Sie:
\begin{equation*}
  \text{a) }5^{40}\bmod3
  \qquad
  \text{b) }(77\cdot34)+(85\cdot44)\bmod4
  \qquad
  \text{c) }2^{3^4}\bmod5
\end{equation*}

% -----------------------------
\paragraph{Übung 11: Aufgabe 6}
% -----------------------------
Beweisen Sie: $(a+b)^5\equiv a^5+b^5\bmod5$ für alle $a,b\in\mathbb{Z}$.

% -----------------------------
\paragraph{Übung 11: Aufgabe 8}
% -----------------------------
Zeigen Sie, dass
\begin{equation*}
  \varphi:(\mathbb{Z},+)\rightarrow(m\mathbb{Z},+)\quad\text{mit}\quad\varphi(x)=mx
\end{equation*}
für $m\in\mathbb{N}\setminus\{0\}$ ein Isomorphismus ist.

% -----------------------------
\paragraph{Übung 12: Aufgabe 4}
% -----------------------------
Beweisen Sie, dass für jeden Homomorphismus $\varphi:G_1\rightarrow G_2$
zwischen zwei endlichen Gruppen folgende Gleichung gilt:
\begin{equation*}
  |G_1|=|\ker(\varphi)|\cdot|\operatorname{im}(\varphi)|
\end{equation*}

% ------------------------------
\subsection{RSA-Verschlüsselung}
% ------------------------------

% ---------------------------
\subsection{Fibonacci-Zahlen}
% ---------------------------

% ------------------------------------------------------------------------------
\end{document}
% ------------------------------------------------------------------------------
